% -*- TeX:UTF-8 -*-
%%
%% KAIST 학위논문양식 LaTeX용 (ver 0.4) 예시		
%%
%% @version 0.4
%% @author  채승병 Chae,Seungbyung (mailto:chess@kaist.ac.kr)
%% @date    2004. 11. 12.
%%
%% @requirement
%% teTeX, fpTeX, teTeX 등의 LaTeX2e 배포판
%% + 은광희 님의 HLaTeX 0.991 이상 버젼 또는 홍석호 님의 HPACK 1.0
%% : 설치에 대한 자세한 정보는 http://www.ktug.or.kr을 참조바랍니다.
%%
%% @note
%% 기존에 널리 쓰여오던 차재춘 님의 학위논문양식 클래스 파일의 형식을
%% 따르지 않고 전면적으로 다시 작성하였습니다. 논문 정보 입력부분에서
%% 과거 양식과 다른 부분이 많으니 아래 예시에 맞춰 바꿔주십시오.
%% 
%%
%% @acknowledgement
%% 본 예시 논문은 물리학과 박사과정 김용현 님의 호의로 제공되었습니다.
%%
%% -------------------------------------------------------------------
%% @information
%% 이 예제 파일은 hangul-ucs를 사용합니다. UTF-8 입력 인코딩으로
%% 작성되었습니다. hlatex의 hfont는 이용하지 않습니다. --2006/02/11

% @class kaist.cls
% @options [default: doctor, korean, final]
% - doctor: 박사과정 | master : 석사과정
% - korean: 한글논문 | english: 영문논문
% - final : 최종판   | draft  : 시험판
% - pdfdoc : 선택하지 않으면 북마크와 colorlink를 만들지 않습니다.
\documentclass[doctor,english,final]{kaist-ucs}
% If you want make pdf document (include bookmark, colorlink)
%\documentclass[doctor,english,final,pdfdoc]{kaist-ucs}

% kaist.cls 에서는 기본으로 dhucs, ifpdf, graphicx 패키지가 로드됩니다.
% 추가로 필요한 패키지가 있다면 주석을 풀고 적어넣으십시오,
%\usepackage{...}

\usepackage{amsmath}

% @command title 논문 제목(title of thesis)
% @options [default: (none)]
% - korean: 한글제목(korean title) | english: 영문제목(english title)
\title[korean] {탄소 나노튜브의 물리적 특성에 대한 이론 연구}
\title[english]{ Theoretical study on physical properties of
                carbon nanotubes}

% @note 표지에 출력되는 제목을 강제로 줄바꿈하려면 \linebreak 을 삽입.
%       \\ 나 \newline 등을 사용하면 안됩니다. (아래는 예시)
%
%\title[korean]{탄소 나노튜브의 물리적 특성에 대한\linebreak 이론 연구}
%\title[english]{Theoretical study on physical properties of\linebreak
%                carbon nanotubes}
%
% If you want to begin a new line in cover, use \linebreak .
% See examples above.
%


% @command author 저자 이름
% @param   family_name, given_name 성, 이름을 구분해서 입력
% @options [default: (none)]
% - korean: 한글이름 | chinese: 한문이름 | english: 영문이름
% 
% If you are a foreigner (this means you have no korean name),
% Write as follow
% \author[korean]{}{}
% \author[chinese]{family name in your native language}{given name in your native language}
% \author[english]{family name in english}{given name in english}
%
\author[korean] {김}{용 현}
\author[chinese]{金}{容 賢}
\author[english]{Kim}{Yong-Hyun}

% @command advisor 지도교수 이름 (복수가능)
% @usage   \advisor[options]{...한글이름...}{...영문이름...}{signed|nosign}
% @options [default: major]
% - major: 주 지도교수  | coopr: 공동 지도교수
\advisor[major]{장 기 주}{Chang, Kee Joo}{signed}
%\advisor[coopr]{홍 길 동}{Gil-Dong Hong}{nosign}
%
% 지도교수 한글이름은 입력하지 않아도 됩니다.
% You may not input advisor's korean name
% like this \advisor[major]{}{Chang, Kee Joo}{signed}
%


% @command department {학과이름}{학위종류} - 아래 표에 따라 코드를 입력
% @command department {department code}{degree field} 
%
% department code table
%
% PH		// 물리학과 Department of Physics 
% MAS	// 수리과학과 Department of Mathematical Sciences
% CH 	// 화학과 Department of Chemistry
% NST	// 나노과학기술대학원 Graduate School of Nanoscience & Technology
% NT		// 나노과학기술 학제전공 Nano Science and Technology Program
% BS  	// 생명과학과 Department of Biological Sciences
% BIS	// 바이오및뇌공학과 Department of Bio and Brain Engineering
% MSE	// 의과학대학원 Graduate School of Medical Science and Engineering
% BM 	// 의과학 학제전공 Biomedical Science and Engineering Program
% CE 	// 건설및환경공학과 Department of Civil and Environmental Engineering
% ME 	// 기계공학전공 Division of Mechanical Engineering
% AE 	// 항공우주공학전공 Division of Aerospace Engineering
% OSE 	// 해양시스템공학전공 Department of Ocean Systems Engineering
% CBE 	// 생명화학공학과 Department of Chemical and Biomolecular Engineering
% MS 	// 신소재공학과 Department of Materials Science and Engineering
% NQE 	// 원자력 및 양자공학과 Department of Nuclear and Quantum Engineering
% EEW 	// EEWS 대학원 Graduate School of EEWS
% PSE 	// 고분자 학제전공 Polymer Science and Engineering Program
% SPE	// 우주탐사공학 학제전공 Space Exploration Engineering Program
% ENY 	// 환경・에너지공학 학제전공 Environmental and Energy Technology Program
% MSB 	// 경영과학과 Department of Management Science
% IT 		// 경영과학과(IT경영학) Department of Management Science (IT Business)
% BAP 	// 경영전문대학원프로그램 Master of Business Administration Program
% ITP 	// 글로벌IT기술대학원프로그램 Global Information & Telecommunication Technology Program
% ITM 	// 기술경영전문대학원 Graduate School of Innovation & Technology Management
% GCT 	// 문화기술대학원 Graduate School of Culture Technology
% CT 	// 문화기술(CT) 학제전공 Culture Technology Program
% EE 	// 전기 및 전자공학과 Department of Electrical Engineering 
% CS 	// 전산학과 Department of Computer Science 
% ICE 	// 정보통신공학과 Department of Information and Communications Engineering 
% IE 		// 산업및시스템공학과 Department of Industrial & Systems Engineering 
% KSE 	// 지식서비스공학과 Department of Knowledge Service Engineering 
% ID 		// 산업디자인학과 Department of Industrial Design 
% RE 	// 로봇공학 학제전공 Robotics Program
% STE 	// 반도체 학제전공 Semiconductor Technology Educational Program 
% SEP 	// 소프트웨어공학프로그램 Software Engineering Program 
% TE 	// 정보통신공학 학제전공 Telecommunication Engineering Program 
% EML 	// e-매뉴팩쳐링리더십 학제전공 e-Manufacturing Leadership Program
% MT 	// 경영공학과 Department of Management Engineering
% TM 	// 테크노경영전공 Techno-MBA Program
% FIN 	// 금융전문대학원 Graduate School of Finance and Accounting
%
% science: 이학 | engineering: 공학 | business : 경영학
% 박사논문의 경우는 학위종류를 입력하지 않아도 됩니다.
% If you write Ph.D. dissertation, you cannot input degree field.

\department{PH}{science}

% @command studentid 학번(ID)
\studentid{20100000}

% @command referee 심사위원 (석사과정 3인, 박사과정 5인)
\referee[1]{장 기 주}
\referee[2]{김 종 진}
\referee[3]{신 성 철}
\referee[4]{신 중 훈}
\referee[5]{이 억 균}
% \referee[5] {Barack Obama}
% Of course english name is available

% @command approvaldate 지도교수논문승인일
% @param   year,month,day 연,월,일 순으로 입력
\approvaldate{2002}{12}{5}

% @command refereedate 심사위원논문심사일
% @param   year,month,day 연,월,일 순으로 입력
\refereedate{2002}{11}{30}

% @command gradyear 졸업년도
\gradyear{2003}

% 본문 시작
\begin{document}

    % 앞표지, 속표지, 학위논문 제출승인서, 학위논문 심사완료 검인서는
    % 클래스 옵션을 final로 지정해주면 자동으로 생성되며,
    % 반대로 옵션을 draft로 지정해주면 생성되지 않습니다.

    % 영문초록 (abstract)
    \begin{abstract}
        For the last decade, carbon nanotubes have been emerging as one
        of ideal materials for the building block of the forthcoming
        nanotechnology, due to their unique electrical and mechanical
        properties. Depending on detailed wrapping-up methods, their
        electronic properties show a wide spectrum from metals to
        large-gap semiconductors with band gaps of 1eV.
        In this thsis, we study various physical properties of carbon
        nanotubes, including electrical properties and their controlling
        methods, magnetic properties, and transport characteristics,
        based on the first-principles density-functional theory and
        the tight-binding model.
    \end{abstract}

    % 목차 (Table of Contents) 생성
    \tableofcontents

    % 표목차 (List of Tables) 생성
    \listoftables

    % 그림목차 (List of Figures) 생성
    \listoffigures

    % 위의 세 종류의 목차는 한꺼번에 다음 명령으로 생성할 수도 있습니다.
    %\makecontents

%% 이하의 본문은 LaTeX 표준 클래스 report 양식에 준하여 작성하시면 됩니다.
%% 하지만 part는 사용하지 못하도록 제거하였으므로, chapter가 문서 내의
%% 최상위 분류 단위가 됩니다.
%% You cannot use 'part'

%\chapter{Introduction}
%
%In 1991, Dr. Iijima at the NEC in Tsukuba published his celebrated
%paper in Nature with the title of ``Helical Microtubules of Graphitic Carbon''
%\cite{Iijima91}.
%It has brought great impact to various fields of physics, chemistry, and
%material science \cite{Dresselhaus96,Saito98}.
%Because of their intrinsic nanometric size, the ``microtubules'' are renamed
%as ``carbon nanotubes'' and can be regarded as ideal materials for realizing
%the forthcoming nanotechnology.
%Depending on the ``helicity', a wide spectrum of electronic structure
%from metal, small-gap (a few meV) to large-gap ($\sim$ 0.5 eV)
%semiconductors is expected.
%People thinks that they can make nano-sized electronic devices like
%field-effect transistors, diodes, and their integrated circuits
%with carbon nanotubes.
%Most of physical properties of carbon nanotubes originates from
%those of ``graphite'': anisotropic electrical properties, strong
%mechanical strength, and flexibility.
\chapter{Introduction}

% simulation study on the modulation of information transfer in feedforward networks


%1 Research Background	1
%1.2 Related Research Trends	 7
%1.3 Research Purpose	9
%1.4 Dissertation Structure
%\section{Computational Neuroscience}
%\section{Realistic Neural Network simulation}

%\section{The NEURON simulator}
\section{Research Background	}
\subsection[The Correlated Neural Activities]{The Correlated Neural Activities; the oscillations and the synchronizations}

Correlated neural activities can be defined in general as, a group of neurons that their firing are not independent to each other, if one neuron fires, the chance of firing of the the others will be either increasing or decreasing~\cite{salinas2001correlated}. The temporally correlated neural activities include oscillations and synchronizations. 

Oscillations are rhythmic activities in the brain that can be observed easily by electroencephalography (EEG) as early as 1930s by Hans Berger, the inventor of EEG~\cite{haas2003hans}. This does not limit only to EEG, the oscillations can be observed by other recording such as local field potential(LFP). 
There are many bands of oscillation frequencies, which are believed to have their particular tasks. For example, alpha band can be found when subjects close their eyes~\cite{buzsaki2004neuronal}. Gamma oscillations are rhythmic activities around 25 - 100Hz with typical frequency at 40 Hz. The gamma oscillation are highly observed in the brain and are believed to be involve in cognitive operations~\cite{engel2001temporal, fries2005mechanism, singer1995visual}. Many models try to explain the mechanisms of gamma oscillation~\cite{brunel2003determines,buzsaki2012mechanisms,gray1996chattering,wang1996gamma,whittington1995synchronized, wilson1972excitatory}. In this work, we are interested in oscillations in gamma frequency.

Synchronization is the phenomena that neurons have the same activities at the same time or in other word they operate in a unison~\cite{ward2003synchronous}. This synchronization can refer to phase synchronization or synchronization of oscillatory~\cite{gray1996chattering, salinas2001correlated,varela2001brainweb}. Similar to oscillations, synchronizations are believed to involve in brain's cognitive process and may be the way of communication in the brain~\cite{engel2001dynamic, fries2001modulation, ward2003synchronous, womelsdorf2007modulation}. Oscillation and Synchronizations are usually observed together. Using sine wave as a metaphor to compare oscillation and synchronization, if the sine wave indicates correlated neural activities then the frequency of the sine wave corresponds to the level of oscillation and the amplitude of the sine wave corresponds to the level of synchronization among neural activities.

Both oscillations and synchronizations are important to the brain. There are many studies suggesting that the unusual level of synchronization and oscillation can result in cognitive dysfunction~\cite{bacsar2008review, dinstein2011disrupted, grice2001disordered, hammond2007pathological, schnitzler2005normal,uhlhaas2006neural, uhlhaas2010abnormal}.



\subsection{The Feedforward Network}

Feedforward network (FFN) is the unidirectional interlayer connection from neurons in lower hierarchy such as sensory neurons to neurons in higher hierarchy such as cortical neurons~\cite{felleman1991distributed, kumar2010spiking}.
The feedforward network is one of the interlayer connections type. The FFN should not be confused with recurrent network - the connection within the same neuronal layer, and the feedback network - the connection from higher hierarchy down to the lower hierarchy that is opposite to FFN~\cite{Bower2003Genesis, carnevale2006neuron, Kandel5thEdition}. 
An example of feedforward connection in a brain is the LGN-V1 connection, as suggested by Hubel and Wiesel~\cite{hubel1962receptive}.

The feedforward connection is important because it is the main path for information to transfer in the brain as signals usually transmit from low-level regions such as sensory input to high-level processing regions such as cortex that analyse such information. 

This work studies the model of feedforward connection in simulation under synchronization.

\section{Related Research Trends}
\subsection{Relationship between synchronization and information in feedforward network (FFN)}

There are many aspects for the relationship between synchronization and information in feedforward network that can be studied. First, there are many studies on the propagation of synchronous spike in simulated feedforward network~\cite{abeles2004modeling,diesmann1999stable, kumar2008conditions}. Diesmann was the first to study the propagation of synchronous activity in computational simulation~\cite{diesmann1999stable}. Similarly, the feedforward synchronization in term of the propagated pattern along retinothalamocortical pathway were studied by analysed multi-unit recording in animal~\cite{neuenschwander2002feed}.
Then, there are some studies on the relation between synchronization and feedforward connection based on theoretical analysis. Goedeke \& Diesmann (2008) explained the synchronization and feedforward network as the relationship between spike intensity to the instantaneous voltage generated by the input~\cite{goedeke2008mechanism}. 
Hahn (2014) showed that oscillatory activity can amplifies synchronous signals in feedforward network~\cite{hahn2014communication}.

This study differs from the above studies in that it focuses on the exact rule of feedforward connection and their relationships among different synchronization levels while they studied only correlations of activities between source and target layer in feedforward connection . This work provides insights on how various feedforward connections rules can be modulated by input synchronization.


\section{Research Purpose}
This work studies how the differences in feedforward connection rules generate different information transfer rate when different degrees of inputs synchronization are given.
It assumes that feedforward convergent connection rules may determine the synchronization dependent response of the network.
\section{Thesis Outline}
This work conducts a simulation study on the modulation of information transfer in feedforward networks, starting with the modelling of generalised model neural network that does not limit to the simulation in this work but can also be applied to various neural systems. It then builds up a model from the generalised model network and discusses a simulation study on the modulation of information transfer in feedforward networks. 






%\section{Old Intro}
%%\subsection{Parameters}
%
%Correlated neural activities such as synchronizations and oscillations are observed in various areas in the brain. A number of studies suggest that this neural synchronization might be a key to understand various brain functions and brain diseases, but the detailed mechanism of how the synchronized neural activity can be systematically controlled is not completely understood yet. 
%In this study, we use the simulated model neural network to understand how the synchronization of spike activities in local neural population can be modulated by the activity of a specific ion channel. Then, we examine the role of synchronization in transmission of information between the neuronal layers. In addition, we examine how different types of interlayer connection affects the level and speed of information transfer across neuronal layers. 
%
%The objectives of the work
%
%o Reproduce the neural network of Parkinson’s Disease animal model
%
%o Study bursting behavior in thalamic layer with and without T-type calcium channel
%
%o Simulate the thalamocortical network of Parkinson’s Disease animal model using statistical wiring diagram
%
%
%The justification for these objectives: Why is the work important?
%
%◦ Existing experiments in animals’ brain require much human effort and time in preparing different types of genetic manipulated animals
%
%◦ Outcomes of most experiments target on only a limited number of species
%
%◦ Results of those experiments are not sufficient to explain neural mechanisms in general
%
%◦ Many simulations use the parameters of experiments from the literatures without considering properties of individual animal species
%◦ Suggests the need to bridge the gap between simulations and experiment effort on animals
%
%Background: Who else has done what? How? What have we done previously?’
%Guidance to the reader: What should the reader watch for in the paper? What are the interesting high points? What strategy did we use?
%Summary/conclusion: What should the reader expect as conclusion?
%Research Question
%o RQ1 : How much can computational simulation resemble experimental results of animal study?
%o RQ2 : Can simulation predict functional connection between thalamus and motor cortex in the real animal?
%o	RQ3 : Is the neural population with higher synchronization level better than the un-synchronized neural population in information transfer to another layer? 
%o	RQ4 : Would the neural population with high synchronization level require small level of convergence input to another layer given the same net level of presynaptic inputs compare to the neural population with unsynchronized activities.
%
%\section{New Intro according to professor on 15.05.29}
%<Re-arrange the wording \& sentences again >
%Mention that you get inspired by the manuscript(?) findings(?) of Dr.Kim that they found the animal with T-type calcium channel get blocked lost the correlation between VL and M1 and resulted in reduce in motor output
%compare to the normal function case . This lead to the question of, what is the role of synchronization in information transfer. In addition, the relationship between the neural synchronization and interlayer connection are not clearly understood. 
%Note : Clearly mentioned that we do not use any of their data in the analysis. Just use it for reference. 
%
%Then talk about basic comp neuro analysis information and technique 
%- Response Function
%- Any others Spike Statistics 



\chapter{Methodology}

(Intro of methodology )


\section{Single Cell Model}

The single cell has been model based on the Hodgkin-Huxley model, the conductance-based single cell model~\cite{hodgkin1952quantitative}.

\begin{align*}
C\frac{dv}{dt} = &-g_L(v - V_L) - G_{Na}(v - V_{Na}) - G_K(v - V_K) - Xg_{CaT}(v - V_{CaT}) \\
&- g_{\sigma E}(t)(v - V_E) - g_{\sigma I}(t)(v - V_I) - g_{input}(t)(v - V_E)
\end{align*}
\begin{align*}
	\text{where,} \hspace{8em} \sigma :& \text{ type of neuron (E or I),}  \\
	g_L :& \text{ leakage conductance,} \\
	g_{\sigma E} \text{ or } g_{\sigma I} :& \text{ synaptic conductance providing E or I input} \\
	C :& \text{ membrane capacitance,}\\
	G_{Na} :& \text{ Na channel conductance,}\\
	G_{K} :& \text{ K channel conductance,}\\
	G_{CaT} :& \text{ T-type Calcium channel conductance,}\\
	X :& \text{ T-type Calcium controlling factor;}\\
	   &  \text{ X = 1 for normal functioning case (WT) }\\
	   &  \text{ X = 0 for not functioning case (KO) }\\
\end{align*}
% comment : add the detail of G (with mnh - parameters)

\subsection{T-Type Calcium Channel}
The  "A model of the T-type calcium current and the low-threshold spike in thalamic neurons" ~\cite{wang1991model}.
-> WT =  add T = T-type functional 
-> KO = no T 

\subsection{Parameter search for single cell model}
Most of the parameters were set with the well-known values. Some parameters - which are 
$g_{CaT}$, $g_{Na}$ - were optimized so that the model single cell shows the same behavior with the experimental results

\subsubsection{Behavior of single cell with and without T-Type calcium channel}

%Comment -> Criteria -> bursting response / tonic response /delta E

\section{Population Model}
\subsection{Model the neuronal mosaics}
 --> Repulsive interaction  // Reference - PIPP model 
 \subsection{Model Baseline activity}
 - Poisson input 
 and 
 - current fluctuation 
 \subsection{Parameter search for population model}
 -  Input - Output function

\section{Connection modelling }

\subsection{Statistical Wiring Diagram}
Connectivity and  Strength of connection depend on distance between cell\cite{ringach2004haphazard,mclaughlin2000neuronal}
\cite{mclaughlin2000neuronal}

\subsection{Lateral connection modelling}


\subsection{Thalamocortical Connection (interlayers connection)}


\subsection{Synchronization and information transfer in the network modelling}


\subsection{Outline}

The experiment: the Parkinson’s disease animal model , WT and KO mice
Network Modeling
o Single cell model with Hodgkin Huxley model and the role of T-Type calcium channel
o Statistical Wiring Diagram
o Local connection modelling
o Thalamocortical connection (Interlayers connection) modelling
o The modelling of MUA and LFP recording
Parameters Search
o Single cell
o Neural Population
o Thalamocortical Connection (interlayers connection)
Synchronization and information transfer in the network modelling
Prediction from the model

First we developed a model neural network that consists of two layers of conductance-based model neurons; the source and the target layer. Then the connection between the two layers are modeled with the statistical wiring rules, where the probability and the strength of connection only depend on the distance between the projection of target neuron on to the source neurons plane. Then we examined the synchronization level in the network while we turn on and off the T-type calcium channel in model neurons in the source layer, which is known to be responsible for generating burst spikes, after a tonic inhibition of membrane potential. Next, we investigate how the variation of synchronization level in source layer contributes differently to the information transfer between the network layers and how the connectivity between layers contributes to level and speed of information transfer between them. 
\chapter{Results}





\section{Prof's suggestion}
\paragraph{}General Note : If you fix trial number to some n value --> explain why it is enough.  
\paragraph{}What are the meaning of these result in biological system
\subsection{Proved your hypothesis}
\paragraph{Hypothesis}  The brain (is it too big or too general?) need interlayer communication that optimized cost of the connection 
which is achieved by the synchronized neural activity on source layer and statistical wiring diagram( with Gaussian distribution of connectivity and connection strength)

\paragraph{Proving the Hypothesis}  Compare differences in these three cases of connection (Gaussian, uniformly random, random with negative exponential) of input variations
 ( osc  vs. no osc  ; varies F and Amp ) 
 The measurement can be  1) Fr ,  2) Spike Correlation, 3) Spike Pattern (Ex. M1 activity according to Input oscillating pattern, how is the phase affect the result etc )

\begin{center}
....................
OLD
....................
\end{center}
\section{ Answer to RQ1 : The computer simulation resemble experimental results of animal study}

\subsection{ The single cell properties of computational model and the whole cell patch clamp recording shows the same behavior}

Figure .1.1: Comparing the sample membrane potential trace of WT and KO during hyperpolarize current injection and depolarize current injection

Figure .1.2: Comparing the number of bursting spikes and tonic spikes between the simulated cell and experimental data


\subsection{  The cell population properties of computational model and the MUA cells recording shows the same behavior}

Figure .1.3 : Comparison of mean and standard deviation of observed firing rate in cell population in computational model and experimental data


\subsection{ Neuron activities during light-off period (no photoactivation)}

Figure .1.4 The baseline activities of neural population during light-off period show no significant different between WT and KO


\subsection{ Neuron activities during photoactivation}

Figure .1.5 The delay and peak of rebound spiking activity of WT and KO are significantly different. The WT shows short rebounding period and higher peak of neural activities


\subsection{ The coherence between VL and M1 layer shows that correlated neural activities after photoactivation ( the rebound bursting spikes) drives activity in M1}

Figure~\ref{fig:sample} The coherence between VL’s MUA and M1’s LFP

\begin{figure}
	\centering
	\includegraphics[width=0.5\textwidth]{figures/sample-fig1}
	\label{fig:sample}
	\caption{Sample Figure}
\end{figure}

\section{Answer to RQ2 : Computational simulation predicts causal relationship of exceed motor command in M1 from VL in Parkinson’s disease patient}

\subsection{ The high synchronization (correlated neural activities) in VL but not average firing rate can drive M1’s motor command}


Figure .2.1 The high synchronized neural activities during bursting can drive M1 in WT types but not KO, even though the average firing rate if WT and KO are not significantly difference


\subsection{  The high synchronization level can be achieved by bursting. The demolishing of bursting in WT neurons result in the absent of synchronization in neuron population}

Figure .2.2 Schematic diagram of how the bursting activity cause high synchronization in neuron population

\subsection{ The Analysis of information transfer from VL to M1 : The information in VL can transfer to M1 when the neuron population in VL are synchronized }

Figure .2.3 the neural information from VL is transferring to M1 only when the neural activities in VL are synchronized.

Figure .2.4 the level of information transfer between VL and M1 is proportionally to the synchronization level in VL

\subsection{ Artificially generated bursting in KO neurons result in high synchronization level of neuron population which can drive M1’s motor command
}
Figure .2.5 successfully generated bursting behavior in KO with similar bursting behavior generated by T-Type Calcium channel in WT

Figure .2.6 The artificial bursting behavior in KO result in high level synchronization level of neural population and this high synchronized neural 
population can drive M1

\subsection{ Artificially activated VL neural population in theta and beta frequencies result in motor command from M1with the same frequency band with what observed in Parkinson’s disease patient.}



%%
%% 표 삽입 예시
%% Example. how to insert table
%%
\begin{table}[t]
\caption{Energy stability $E$ (eV) per molecule of all meta-stable
isomer states of C$_{60}$ opening process for forming the (5,5) cap.
In the SW-I and SW-II, both ferromagnetic (Ferro) and paramagnetic (Para)
spin configurations are obtained, whereas only non-magnetic configuration
is obtained in the BF, SW-III, and CAP(5,5).
$M$ is total magnetization $n_{\rm up}$-$n_{\rm down}$ in unit of $\mu_B$, where
$n_{\rm up(down)}$ is the number of up (down) spins.
}
\label{mag-tab1}
\begin{center}
\begin{tabular} {ccccccccccc}
\hline\hline
& & BF &\multicolumn{2}{c}{SW-I}&&\multicolumn{2}{c}{SW-II}&SW-III&CAP&\\
\cline{4-5} \cline{7-8}
&               &   &  Para & Ferro &&   Para &  Ferro &      &      &\\
\hline
& $E$ (eV)      & 0 & 7.796 & 7.832 && 10.418 & 10.408 & 11.5 & 13.2 &\\
& $M$ ($\mu_B$) & 0 &     0 &  1.94 &&      0 &   2.06 &    0 &    0 &\\
\hline\hline
\end{tabular}
\end{center}
\end{table}


%%
%% 그림 삽입 예시
%% Example. how to insert graph
%%
%% Note. 가급적 \includegraphics 명령을 사용하십시오.
%% Recommen : Use \includegraphics to insert graph.
%%

\chapter{Conclusion}

What does it mean?
Successfully regenerate experimental results with computational simulation
The simulation shows functional connection between thalamus and motor cortex in the real animal
What hypotheses were proved or disproved?
We can make computational simulation which resemble the experimental data
The T-Type calcium channel generate bursting behavior of single cell
The neural population are highly synchronized during bursting behavior
The high synchronization level in neural population can transfer information from one layer to another layer
What did I learn?
I can make computational model that can regenerate experimental data and I can use it to predict new properties of neural system
Why does it make a differences?
The simulation predicts functional connection between VL and M1 neuronal layers
The simulation suggest that reverse testing of KO cells to resemble WT can also drive motor command in M1. The finding suggests that the bursting is the important factor for high synchronization level of neural population and it is the key for neural network to transfer data from VL to M1

Add a new, higher level of analysis
Indicate explicitly the significance of the work
This work shows the potential of using computational simulation to regenerate experimental data in silico and employ it to manipulate properties of neuron network that are hard to do in the experiments and use it to predict new hypothesis


%%
%% 참고문헌 시작
%% References
%%

%\begin{thebibliography}{00}
%
%\bibitem{Iijima91} S. Iijima,
%         Nature (London) {\bf 354}, 56 (1991).
%         %Helical microtubules of graphitic carbon.
%
%\bibitem{Dresselhaus96} M. S. Dresselhaus, G. Dresselhaus, and P. C. Eklund.
%         {\em Science of Fullerenes and Carbon Nanotubes} (Academic, San Diego, 1996).
%
%\bibitem{Saito98} R. Saito, G. Dresselhaus, and M. S. Dresselhaus,
%         {\em Physical Properties of Carbon Nanotubes}
%         (Imperial College Press, London, 1998).
%
%\bibitem{Makarova01} T.L. Makarova, B. Sundqvist,
%         R. H\"ohne, P. Esquinazi, Y. Kopelevich, P. Scharff,
%         V.A. Davydov, L.S. Kashevarova, and A.V. Rakhmanina,
%         Nature (London) {\bf 413}, 716 (2001).
%
%\bibitem{Palacio01} F. Palacio, Nature (London) {\bf 413}, 690 (2001),
%         and references therein.
%
%\bibitem{SW} A.J. Stone and D.J. Wales,
%         Chem. Phys. Lett {\bf 128}, 501 (1986).
%
%\end{thebibliography}

\bibliographystyle{naturemag}
\bibliography{references}

%%
%% 한글요약문 시작 (Korean summary)
%%
%% Note. 영문논문일 경우에만 필요하니 한글논문의 경우에는 작성하지 마십시오.
%%
\begin{summary}

    지난 10여 년간 탄소 나노튜브는 자체의 독특한 전기적, 기계적 성질로
    인하여 다가오는 나노기술 분야의 이상적인 기초물질중의 하나로 떠오르고
    있다. 흑연을 감는 세세한 방법에 따라 전기적 특성이 금속성에서 1eV의
    띠간격을 가지는 반도체 특성까지 다양한 분포로 존재한다.
    본 학위논문에서는 탄소 나노튜브의 여러 물리적 성질에 대해 고찰하는데,
    기본적으로 제일원리 밀도함수 이론과 밀접결합근사 모형을 사용하여 전기적
    특성과 그 제어 방법, 자기적 특성, 그리고 수송특성 등을 다루고자 한다.

\end{summary}

%%
%% 감사의 글 시작
%% Acknowledgement
%%
% @command acknowledgement 감사의글
% @options [default: 클래스 옵션 korean|english ]
% - korean : 한글타이틀 | english : 영문타이틀

\acknowledgement[korean]

    이 논문을 완성하기까지 주위의 모든 분들로부터 수많은 도움을 받았습니다.

    끝으로 오늘의 제가 있을 수 있도록 사랑으로 키워 주신
    어머니와 또한 가족들에게 감사드립니다.
    저의 이 작은 결실이 그분들께 조금이나마 보답이 되기를 바랍니다.

%%
%% 이력서 시작
%% Curriculum Vitae
%%
% @command curriculumvitae 이력서
% @options [default: 클래스 옵션 korean|english ]
% - korean : 한글이력서 | english : 영문이력서
\curriculumvitae[korean]

    % @environment personaldata 개인정보
    % @command     name         이름
    %              dateofbirth  생년월일
    %              birthplace   출생지
    %              domicile     본적지
    %              address      주소지
    %              email        E-mail 주소
    % - 위 6개의 기본 필드 중에 이력서에 적고 싶은 정보를 입력
    % input data only you want
    \begin{personaldata}
        \name       {김 용 현}
        \dateofbirth{1972}{3}{25}
        \birthplace {대전 대덕구 평촌동 ...}
        \domicile   {대전 유성구 신성동 ...}
        \address    {대전 유성구 구성동 ...}
        \email      {yonghyunkim@test.kaist.ac.kr}
    \end{personaldata}

    % @environment education 학력
    % @options [default: (none)] - 수학기간을 입력
    \begin{education}
        \item[1988. 3.\ --\ 1990. 2.] 대전과학고등학교 (2년 수료)
        \item[1990. 3.\ --\ 1997. 2.] 한국과학기술원 물리학과 (B.S.)
        \item[1997. 3.\ --\ 1999. 2.] 한국과학기술원 물리학과 (M.S.)
    \end{education}

    % @environment career 경력
    % @options [default: (none)] - 해당기간을 입력
    \begin{career}
        \item[1997. 3.\ --\ 1999. 2.] 한국과학기술원 물리학과 일반조교
        \item[2001. 7.\ --\ 2002. 1.] 한국과학기술정보연구원 슈퍼컴퓨팅센터 위촉연구원
        \item[2002. 8.\ --\ 2002. 12.] 한국표준과학연구원 연구생
    \end{career}

    % @environment activity 학회활동
    % @options [default: (none)] - 활동내용을 입력
    \begin{activity}
        \item J. Choi, \textbf{Yong-Hyun Kim}, K.J. Chang, and D. Tomanek,
             \textit{Occurrence of itinerant ferromagnetism in C/BN superlattice
             nanotubes}, 5th Asian Workshop on First-Principles Electronic
             Structure Calculations, Seoul (Korea), October., 2002.
    \end{activity}

    % @environment publication 연구업적
    % @options [default: (none)] - 출판내용을 입력
    \begin{publication}
        \item \textbf{Yong-Hyun Kim}, J. Choi, K.J. Chang, and D. Tomanek,
             \textit{Magnetic instability in partly opened C$_{60}$ isomers},
             in preparation.
    \end{publication}

%% 본문 끝
\end{document}
