\chapter{Introduction}

\section{Old Intro}
%\subsection{Parameters}

Correlated neural activities such as synchronizations and oscillations are observed in various areas in the brain. A number of studies suggest that this neural synchronization might be a key to understand various brain functions and brain diseases, but the detailed mechanism of how the synchronized neural activity can be systematically controlled is not completely understood yet. 
In this study, we use the simulated model neural network to understand how the synchronization of spike activities in local neural population can be modulated by the activity of a specific ion channel. Then, we examine the role of synchronization in transmission of information between the neuronal layers. In addition, we examine how different types of interlayer connection affects the level and speed of information transfer across neuronal layers. 

The objectives of the work

o Reproduce the neural network of Parkinson’s Disease animal model

o Study bursting behavior in thalamic layer with and without T-type calcium channel

o Simulate the thalamocortical network of Parkinson’s Disease animal model using statistical wiring diagram


The justification for these objectives: Why is the work important?

◦ Existing experiments in animals’ brain require much human effort and time in preparing different types of genetic manipulated animals

◦ Outcomes of most experiments target on only a limited number of species

◦ Results of those experiments are not sufficient to explain neural mechanisms in general

◦ Many simulations use the parameters of experiments from the literatures without considering properties of individual animal species
◦ Suggests the need to bridge the gap between simulations and experiment effort on animals

Background: Who else has done what? How? What have we done previously?’
Guidance to the reader: What should the reader watch for in the paper? What are the interesting high points? What strategy did we use?
Summary/conclusion: What should the reader expect as conclusion?
Research Question
o RQ1 : How much can computational simulation resemble experimental results of animal study?
o RQ2 : Can simulation predict functional connection between thalamus and motor cortex in the real animal?
o	RQ3 : Is the neural population with higher synchronization level better than the un-synchronized neural population in information transfer to another layer? 
o	RQ4 : Would the  neural population with high synchronization level  require small level of convergence input to another layer given the same net level of presynaptic inputs compare to the neural population with unsynchronized activities.

\section{New Intro according to professor on 15.05.29}
<Re-arrange the wording \& sentences again >
Mention that you get inspired by the manuscript(?) findings(?) of Dr.Kim that they found the animal with T-type calcium channel get blocked lost the correlation between VL and M1 and resulted in reduce in motor output
compare to the normal function case . This lead to the question of, what is the role of synchronization in information transfer.  In addition, the relationship between the neural synchronization and interlayer connection are not clearly understood. 
Note : Clearly mentioned that we do not use any of their data in the analysis. Just use it for reference. 

Then talk about basic comp neuro analysis information and technique 
- Response Function
- Any others  Spike Statistics 
\section{Computational Neuroscience}
\section{Realistic Neural Network simulation}

\section{The NEURON simulator}
\section{The Correlated Neural Activities; the oscillation and synchronization}
\section{The Feedforward Network}