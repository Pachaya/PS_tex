\chapter{Introduction}

% simulation study on the modulation of information transfer in feedforward networks


%1 Research Background	1
%1.2 Related Research Trends	 7
%1.3 Research Purpose	9
%1.4 Dissertation Structure
%\section{Computational Neuroscience}
%\section{Realistic Neural Network simulation}

%\section{The NEURON simulator}
\section{Research Background	}
\subsection[The Correlated Neural Activities]{The Correlated Neural Activities; the oscillations and the synchronizations}

Correlated neural activities can be defined in general as, a group of neurons that their firing are not independent to each other, if one neuron fires, the chance of firing of the the others will be either increasing or decreasing~\cite{salinas2001correlated}. The temporally correlated neural activities include oscillations and synchronizations. 

Oscillations are rhythmic activities in the brain that can be observed easily by electroencephalography (EEG) as early as 1930s by Hans Berger, the inventor of EEG~\cite{haas2003hans}. This does not limit only to EEG, the oscillations can be observed by other recording such as local field potential(LFP). 
There are many bands of oscillation frequencies, which are believed to have their particular tasks. For example, alpha band can be found when subjects close their eyes~\cite{buzsaki2004neuronal}. Gamma oscillations are rhythmic activities around 25 - 100Hz with typical frequency at 40 Hz. The gamma oscillation are highly observed in the brain and are believed to be involve in cognitive operations~\cite{engel2001temporal, fries2005mechanism, singer1995visual}. Many models try to explain the mechanisms of gamma oscillation~\cite{brunel2003determines,buzsaki2012mechanisms,gray1996chattering,wang1996gamma,whittington1995synchronized, wilson1972excitatory}. In this work, we are interested in oscillations in gamma frequency.

Synchronization is the phenomena that neurons have the same activities at the same time or in other word they operate in a unison~\cite{ward2003synchronous}. This synchronization can refer to phase synchronization or synchronization of oscillatory~\cite{gray1996chattering, salinas2001correlated,varela2001brainweb}. Similar to oscillations, synchronizations are believed to involve in brain's cognitive process and may be the way of communication in the brain~\cite{engel2001dynamic, fries2001modulation, ward2003synchronous, womelsdorf2007modulation}. Oscillation and Synchronizations are usually observed together. Using sine wave as a metaphor to compare oscillation and synchronization, if the sine wave indicates correlated neural activities then the frequency of the sine wave corresponds to the level of oscillation and the amplitude of the sine wave corresponds to the level of synchronization among neural activities.

Both oscillations and synchronizations are important to the brain. There are many studies suggesting that the unusual level of synchronization and oscillation can result in cognitive dysfunction~\cite{bacsar2008review, dinstein2011disrupted, grice2001disordered, hammond2007pathological, schnitzler2005normal,uhlhaas2006neural, uhlhaas2010abnormal}.



\subsection{The Feedforward Network}

Feedforward network (FFN) is the unidirectional interlayer connection from neurons in lower hierarchy such as sensory neurons to neurons in higher hierarchy such as cortical neurons~\cite{felleman1991distributed, kumar2010spiking}.
The feedforward network is one of the interlayer connections type. The FFN should not be confused with recurrent network - the connection within the same neuronal layer, and the feedback network - the connection from higher hierarchy down to the lower hierarchy that is opposite to FFN~\cite{Bower2003Genesis, carnevale2006neuron, Kandel5thEdition}. 
An example of feedforward connection in a brain is the LGN-V1 connection, as suggested by Hubel and Wiesel~\cite{hubel1962receptive}.

The feedforward connection is important because it is the main path for information to transfer in the brain as signals usually transmit from low-level regions such as sensory input to high-level processing regions such as cortex that analyse such information. 

This work studies the model of feedforward connection in simulation under synchronization.

\section{Related Research Trends}
\subsection{Relationship between synchronization and information in feedforward network (FFN)}

There are many aspects for the relationship between synchronization and information in feedforward network that can be studied. First, there are many studies on the propagation of synchronous spike in simulated feedforward network~\cite{abeles2004modeling,diesmann1999stable, kumar2008conditions}. Diesmann was the first to study the propagation of synchronous activity in computational simulation~\cite{diesmann1999stable}. Similarly, the feedforward synchronization in term of the propagated pattern along retinothalamocortical pathway were studied by analysed multi-unit recording in animal~\cite{neuenschwander2002feed}.
Then, there are some studies on the relation between synchronization and feedforward connection based on theoretical analysis. Goedeke \& Diesmann (2008) explained the synchronization and feedforward network as the relationship between spike intensity to the instantaneous voltage generated by the input~\cite{goedeke2008mechanism}. 
Hahn (2014) showed that oscillatory activity can amplifies synchronous signals in feedforward network~\cite{hahn2014communication}.

This study differs from the above studies in that it focuses on the exact rule of feedforward connection and their relationships among different synchronization levels while they studied only correlations of activities between source and target layer in feedforward connection . This work provides insights on how various feedforward connections rules can be modulated by input synchronization.


\section{Research Purpose}
This work studies how the differences in feedforward connection rules generate different information transfer rate when different degrees of inputs synchronization are given.
It assumes that feedforward convergent connection rules may determine the synchronization dependent response of the network.
\section{Thesis Outline}
This work conducts a simulation study on the modulation of information transfer in feedforward networks, starting with the modelling of generalised model neural network that does not limit to the simulation in this work but can also be applied to various neural systems. It then builds up a model from the generalised model network and discusses a simulation study on the modulation of information transfer in feedforward networks. 






%\section{Old Intro}
%%\subsection{Parameters}
%
%Correlated neural activities such as synchronizations and oscillations are observed in various areas in the brain. A number of studies suggest that this neural synchronization might be a key to understand various brain functions and brain diseases, but the detailed mechanism of how the synchronized neural activity can be systematically controlled is not completely understood yet. 
%In this study, we use the simulated model neural network to understand how the synchronization of spike activities in local neural population can be modulated by the activity of a specific ion channel. Then, we examine the role of synchronization in transmission of information between the neuronal layers. In addition, we examine how different types of interlayer connection affects the level and speed of information transfer across neuronal layers. 
%
%The objectives of the work
%
%o Reproduce the neural network of Parkinson’s Disease animal model
%
%o Study bursting behavior in thalamic layer with and without T-type calcium channel
%
%o Simulate the thalamocortical network of Parkinson’s Disease animal model using statistical wiring diagram
%
%
%The justification for these objectives: Why is the work important?
%
%◦ Existing experiments in animals’ brain require much human effort and time in preparing different types of genetic manipulated animals
%
%◦ Outcomes of most experiments target on only a limited number of species
%
%◦ Results of those experiments are not sufficient to explain neural mechanisms in general
%
%◦ Many simulations use the parameters of experiments from the literatures without considering properties of individual animal species
%◦ Suggests the need to bridge the gap between simulations and experiment effort on animals
%
%Background: Who else has done what? How? What have we done previously?’
%Guidance to the reader: What should the reader watch for in the paper? What are the interesting high points? What strategy did we use?
%Summary/conclusion: What should the reader expect as conclusion?
%Research Question
%o RQ1 : How much can computational simulation resemble experimental results of animal study?
%o RQ2 : Can simulation predict functional connection between thalamus and motor cortex in the real animal?
%o	RQ3 : Is the neural population with higher synchronization level better than the un-synchronized neural population in information transfer to another layer? 
%o	RQ4 : Would the neural population with high synchronization level require small level of convergence input to another layer given the same net level of presynaptic inputs compare to the neural population with unsynchronized activities.
%
%\section{New Intro according to professor on 15.05.29}
%<Re-arrange the wording \& sentences again >
%Mention that you get inspired by the manuscript(?) findings(?) of Dr.Kim that they found the animal with T-type calcium channel get blocked lost the correlation between VL and M1 and resulted in reduce in motor output
%compare to the normal function case . This lead to the question of, what is the role of synchronization in information transfer. In addition, the relationship between the neural synchronization and interlayer connection are not clearly understood. 
%Note : Clearly mentioned that we do not use any of their data in the analysis. Just use it for reference. 
%
%Then talk about basic comp neuro analysis information and technique 
%- Response Function
%- Any others Spike Statistics 
