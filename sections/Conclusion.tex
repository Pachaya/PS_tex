\chapter{Conclusion}

This work introduced the generalised model for neural network simulation.  The model acts as a template for simulation study that does not limited to only this paper, but also any other neural network models. Using this generalised model, a model for study on the modulation of information transfer in feedforward network was made. 
In order to study how information transfer can be modulated in the feedforward networks, three levels of input synchronization were given to the target layer under three convergent connection rules (GG, UU, UE).  In this work, we studied the modulation of information transfer by mean of the output firing rate that appears when input changes from static to oscillating input.

First, we described a unique activity map under various types of input and various types of convergent connections rules. We then showed an increase in the output firing rate when the feedforward connection strength increases and when the synchronization level of the input increases. We demonstrated that the Uniform-Uniform convergent rule (UU) has the highest response gain from a static input to an oscillating input when compared to other convergent rules. Finally, we revealed that the UU rule has the highest induced spike and that it is highly dependent on the phase of the input oscillation. An explanation of why the UU rule is the most sensitive to synchronized input based on the idea of temporal summation was also given.


In summary, we found that high level of synchronization in input results in high output response and the synchronized input can make a specific level of output with low cost of connection compared to the static input. In addition, neural network with Uniform-Uniform convergent connection rule is selective to change in input synchronization compared to the other two rules. These results suggest that both input synchronization level and the interlayer connection rules can contribute to the understanding of brain connections. 

\paragraph{} 

%What does it mean?
%Successfully regenerate experimental results with computational simulation
%The simulation shows functional connection between thalamus and motor cortex in the real animal
%What hypotheses were proved or disproved?
%We can make computational simulation which resemble the experimental data
%The T-Type calcium channel generate bursting behavior of single cell
%The neural population are highly synchronized during bursting behavior
%The high synchronization level in neural population can transfer information from one layer to another layer
%What did I learn?
%I can make computational model that can regenerate experimental data and I can use it to predict new properties of neural system
%Why does it make a differences?
%The simulation predicts functional connection between VL and M1 neuronal layers
%The simulation suggest that reverse testing of KO cells to resemble WT can also drive motor command in M1. The finding suggests that the bursting is the important factor for high synchronization level of neural population and it is the key for neural network to transfer data from VL to M1
%
%Add a new, higher level of analysis
%Indicate explicitly the significance of the work
%This work shows the potential of using computational simulation to regenerate experimental data in silico and employ it to manipulate properties of neuron network that are hard to do in the experiments and use it to predict new hypothesis
